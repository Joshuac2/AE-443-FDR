Utilizing widespread market comparison studies of eight contemporary aircraft closely fitting the RFP \cite{RFP} requirement range and capacity from different manufacturers, eras, and continents, the team was able to formulate an initial design fulfilling or exceeding the requirements for a 400 passenger, short to medium haul aircraft. Key decisions have been made to utilize newfound technological advantages in composites and high-lift devices in order to improve the characteristics of the aircraft of this size for short haul (hundreds of nmi) flights.  The steps towards iteratively sizing and measuring the performance of this aircraft have been made, as has a foundation for strong justification of team decisions regarding the final detailed design of the aircraft.  Additionally, significant strides have be made defining the aircraft's hydraulic, electrical, and environmental control systems, including adaptation of a state of the art avionics suite fulfilling and exceeding the autopilot requirements set forth by AIAA \cite{RFP}.

A considerable amount of work remains in regards to further optimization of the aircraft's subsystems, with special attention regarding the wing and structure in order to fulfil the team goal of designing the best possible high capacity, short range, low budget aircraft. Further definition in CAD and proving in ANSYS remain, however these are the final steps in developing the aircraft's farther extremities.  Additionally, continuing verification of FAA CFR Part 25 \cite{cfr} requirements throughout all aspects of the aircraft remain.  By the conclusion of the final report, Toucan's Sam Mark I will have a build up analysis of each major system outlined in this report, as well as an estimation of production timelines and both initial and service-life maintenance cost.  Throughout this project, the team at Toucan has strode to develop a value-driven aircraft capable of exceeding client expectations while solving one of industry's toughest and omnipresent problems by applying a blend of new and proven technology and methodology to the design of the Toucan SAM Mark I presented in this report.  



% \textcolor{red}{
% \begin{itemize}
%     \item Summarize your motivation, design objectives, proposed design, key performance metrics, and key differentiators. \checkmark (\textit{JC})
%     \item Discuss problems encountered (e.g. requirements not met) and recommendations for future study (if any). \checkmark (\textit{JC})
%     \item Summarize future work. \checkmark (\textit{JC})
% \end{itemize}}

%Finally, significant strides will be made defining the aircraft's hydraulic, electrical, and environmental control systems, including adaptation of a state of the art avionics suite fulfilling the autopilot requirements set forth by AIAA.